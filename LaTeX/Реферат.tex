\abstract{РЕФЕРАТ}

Объем работы равен \formbytotal{lastpage}{страниц}{е}{ам}{ам}. Работа содержит \formbytotal{figurecnt}{иллюстраци}{ю}{и}{й}, \formbytotal{tablecnt}{таблиц}{у}{ы}{}, \arabic{bibcount} библиографических источников. Количество приложений – 2. Графический материал представлен в приложении А. Фрагменты исходного кода представлены в приложении Б.

Перечень ключевых слов: Портфолио фотографа, веб-сайт, фотографии высокого разрешения, галерея работ, описание работ, дизайн сайта, контактная информация, изображения в высоком качестве, визуальный стиль, оптимизация для мобильных устройств, структура портфолио

Объектом разработки является веб-сайт фотографа, представляющий его портфолио, визуализируя и подчеркивая его творческий подход и профессиональные навыки.

Целью выпускной квалификационной работы является привлечение клиентов, увеличение популярности фотографа, информирование о стиле и подходе фотографа к своей работе.

В процессе разработки веб-сайта были выделены следующие ключевые элементы: галерея высококачественных изображений работ фотографа, описание к каждой фотографии или серии фотографий, оригинальный и творческий визуальный стиль, отражающий стиль фотографа, навигация, использование современной системы управления контентом для обеспечения легкости в администрировании и обновлении контента

Разработанный веб-сайт успешно внедрен и служит эффективным инструментом для привлечения клиентов и демонстрации профессионализма фотографа.

\selectlanguage{english}
\abstract{ABSTRACT}
  
The volume of work is \formbytotal{lastpage}{page}{}{s}{s}. The work contains \formbytotal{figurecnt}{illustration}{}{s}{s}, \formbytotal{tablecnt}{table}{}{s}{s}, \arabic{bibcount} bibliographic sources and \formbytotal{числоПлакатов}{sheet}{}{s}{s} of graphic material. The number of applications is 2. The graphic material is presented in annex A. The layout of the site, including the connection of components, is presented in annex B.

List of key words: Photographer's portfolio, website, high-resolution photos, gallery of works, work descriptions, website design, contact information, high-quality images, visual style, optimization for mobile devices, portfolio structure.


The object of development is a photographer's website, presenting their portfolio, visualizing and highlighting their creative approach and professional skills.

The goal of the final qualifying work is to attract clients, increase the photographer's popularity, and inform about the photographer's style and approach to their work.

During the development of the website, the following key elements were identified: a gallery of high-quality images showcasing the photographer's works, descriptions for each photograph or series of photographs, an original and creative visual style reflecting the photographer's unique style, navigation, and the use of a modern content management system to ensure ease of administration and content updates.

The developed website has been successfully implemented and serves as an effective tool for attracting clients and showcasing the photographer's professionalism.
\selectlanguage{russian}
