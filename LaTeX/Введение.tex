\section*{ВВЕДЕНИЕ}
\addcontentsline{toc}{section}{ВВЕДЕНИЕ}
Фотография, как искусство, переживает новый виток развития, и виртуозные работы фотографов становятся неотъемлемой частью этой эпохи. В современном цифровом мире, где каждое мгновение можно запечатлеть, веб-сайт портфолио фотографа выступает важным инструментом для демонстрации его творческого потенциала и профессиональных навыков.

Наш веб-сайт представляет собой виртуальную выставку, где каждое изображение – это не просто фотография, а произведение искусства. Мы стремимся подчеркнуть индивидуальность и талант каждого фотографа через высококачественные фотографии, оригинальный дизайн и легкость навигации.

Цель нашего веб-сайта – не только предоставить вам возможность насладиться красотой фотографий, но и сделать ваш выбор проще, предоставив подробные описания каждой работы. Мы стремимся создать уникальное визуальное пространство, которое погружает посетителя в мир фотографии.

В процессе разработки веб-сайта мы уделили особое внимание галерее высококачественных изображений, описанию каждой работы, созданию оригинального визуального стиля и использованию современных технологий для обеспечения удобства взаимодействия с нашим контентом.

\emph{Цель настоящей работы} – разработка web-сайта портфолио фотографа для привлечения новой аудитории, увеличения заказов и услуг компании. Для достижения поставленной цели необходимо решить \emph{следующие задачи:}
\begin{itemize}
\item провести анализ предметной области;
\item разработать концептуальную модель web-сайта;
\item спроектировать web-сайт;
\item реализовать сайт средствами web-технологий.
\end{itemize}

\emph{Структура и объем работы.} Отчет состоит из введения, 4 разделов основной части, заключения, списка использованных источников, 2 приложений. Текст выпускной квалификационной работы равен \formbytotal{page}{страниц}{е}{ам}{ам}.


\emph{Во введении} мы ставим перед собой задачу создания веб-портфолио, которое не только подчеркнет уникальный стиль и творческий подход фотографа, но и станет эффективным инструментом для привлечения новых клиентов. Мы расскажем о целях работы, опишем структуру веб-сайта и предоставим краткое содержание каждого из его разделов.

\emph{В первом разделе} на этапе анализа предметной области мы предпримем шаг в мир фотографии, изучив стиль, особенности и основные работы нашего фотографа. Это позволит нам лучше понять требования к веб-портфолио.

\emph{Во втором разделе} на этапе технического задания мы выявим основные требования к веб-сайту. Здесь мы определим, какие функциональности будут доступны посетителям, как обеспечим удобство навигации, и какие дополнительные элементы дизайна мы внедрим.

\emph{В третьем разделе} на этапе технического проектирования мы представим концептуальные идеи для дизайна веб-портфолио. Здесь мы обсудим, каким образом выделить каждую фотографию, как организовать галерею и каким образом донести творческую индивидуальность фотографа через визуальный стиль.

\emph{В четвертом разделе} мы представим список ключевых компонентов веб-портфолио и их функциональные возможности. Кроме того, мы расскажем о процессе тестирования, который гарантирует корректную работу веб-сайта и соответствие всем требованиям.

В заключении излагаются основные результаты работы, полученные в ходе разработки.

В приложении А представлен графический материал.
В приложении Б представлены фрагменты исходного кода. 
